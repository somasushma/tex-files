\documentclass[12pt]{article}
\usepackage[margin=1.3in]{geometry}
\usepackage{fontspec}
\usepackage[english]{babel} 
\newfontfamily\skt[Script=Devanagari]{Adobe Devanagari}
\newfontfamily\sktr[Script=Devanagari]{Adishila}
\usepackage{parskip}
\usepackage{float}
\usepackage{wrapfig}
\usepackage{array}
\usepackage{amsmath}
\usepackage{amssymb}
\usepackage{array}
\usepackage[dvipsnames]{xcolor}
\usepackage{graphicx}
\graphicspath{ {images/} }
\usepackage{tikz}
\usepackage{tikz-network}
\usetikzlibrary{arrows, arrows.meta, patterns, shapes.geometric, shapes.misc, graphs, mindmap, calc}
\title{\textbf{ {\color{Sepia}{\skt ॥ पूर्व-कामिकागमे महाकवचम् ॥ }}}}
\author{}
\date{}
\begin{document}
	\maketitle
\begin{center}
	\tikzset{
		petal3/.pic={
			\draw[fill=#1] (0,0) .. controls (0,3) and (1.5,0) .. (2,0) .. controls (1.5,0) and (0,-3) .. (0,0) -- cycle;
		}
	}
	\begin{tikzpicture}
	%petals
		\foreach \x in {0,45,90, 135,180, 225,270, 315}{
		\draw[Fuchsia, fill=Fuchsia!30!White, line width=.8] (\x:2.155) pic[scale=0.74, rotate=\x] {petal3};
	}
 %circles 
	\filldraw[line width=1, Blue] (0,0) circle(2);
	\draw[line width=1, Red] (0,0) circle(2.05);
	\draw[line width=1, Red] (0,0) circle(2.1);
	\draw[line width=1, Red] (0,0) circle(2.15);
%triangles
	\draw[line width=1, Goldenrod] (270:2) -- (30:2) -- (150:2) -- cycle;
    \draw[line width=1, Goldenrod] (0,1) -- (1.4012585384, -1.4270509831) -- (-1.401259, -1.427051) -- cycle;
    \draw[line width=1, Goldenrod] (-0.866025403784891, -0.5) -- (1.93649167310365, -0.5) -- (0.535233134659599, 1.92705098312447) -- cycle;
    \draw[line width=1, Goldenrod] (-1.93649167310365, -0.5) -- (0.866025403784891, -0.5) -- (-0.535233134659599, 1.92705098312447) -- cycle;
    \node[Goldenrod] (s1) at (0,0.1) {\large{\sktr{ह्सौं}}};
 %doors
\draw[Periwinkle, rounded corners=3pt, line width=1.5, double] (.5,-4.5) -- (.5, -4) -- (4, -4) -- (4,-.5) -- (4.5, -.5);
\foreach \x / \y in {-1/1, 1/-1, -1/-1}{
	\begin{scope}[xscale=\x, yscale=\y]
	\draw[Periwinkle, rounded corners=3pt, line width=1.5, double] (.5,-4.5) -- (.5, -4) -- (4, -4) -- (4,-.5) -- (4.5, -.5);
	\end{scope}}
\end{tikzpicture}\\	

\end{center}
{\color{NavyBlue}{\large{\sktr
ॐ व्योम-व्यापिने दिव्यान्तरिक्ष-रूपाय सर्वव्यापिने प्रधानाय ॐ नमः शिवाय महादेवाय व्योमाष्ट-भासायानेकाकृति-त्रिरूपाय पदपदाय सर्वविघ्न-विनाशाय रौद्ररूपाय विशेषातीताय भिलि पिलि किंशिलि लिपि वक्त्र-दंष्ट्राय भयं-कृत-नयनायोपलेलिहानोग्रजिह्वाय ऋग्-यजुः-सामाथर्वेन्द्रियाय + अनन्ताय + अनाभाय + अनाश्रिताय ध्रुवाय शाश्वताय किल-किल लम्ब योगपीठसंस्थिताय नित्यं योगिने ध्यानाहाराय आधाराय धेयाय ईशानाय जगत्पतये शिवाय तत्पुरुषवक्त्राय नमो घोरशिवायाष्टहृदयाय शिवकवचेन्द्राय विद्युत्प्रपन्नाय वामदेवगुह्याय पचपचाय वरवरेण्याय परिसुवीरानन्दसद्योजात-मूर्तये वामादिशक्त्याधाराय कालघ्नकामापहाराय बल-बल-मनोन्मन-वामा-ज्येष्ठा-रुद्राय शक्तिगुह्याय बहुरूपिणे नमः ॥\\[6pt]
शब्दाय शब्दरूपिणे स्वर्गाय स्वर्गरूपिणे तेजसे तेजोरूपिणे रसाय रसरूपिणे गन्धाय गन्धरूपिणे बीजाय बीजरूपिणे मूलाय मूलरूपिणे अङ्कुराय अङ्कुररूपिणे नालाय नालरूपिणे पत्राय पत्ररूपिणे गर्भाय गर्भरूपिणे पुष्पाय पुष्परूपिणे फलाय फलरूपिणे अनन्ताय अनन्तरूपिणे रेतसे रेतोरूपिणे मनसे मनोरूपिणे अर्थाय अर्थरूपिणे अहंकाराय अहंकाररूपिणे प्रकृतिरूपाय प्रकृतिरूपिणे मायाय मायारूपिणे शुद्धाय शुद्ध-रूपिणे विद्याय विद्यारूपिणे नयाय नयरूपिणे सदाशिवाय सदाशिवरूपिणे ॐ हौं शिवाय नमः ॥\\[6pt] 
कालाग्निरुद्राय मूर्त्यष्टक-विभूषिताय नित्यानित्याय नमो हिरण्यबाहवे अनिलाय अनिलवर्चसे हेतुकारणाय नमः | त्रिपुर-त्रिपुरातीत-शान्त-शान्तातीत-घोरघोरतरः शिवतमोऽवताच्छिवतमः | अनन्तादिनागाय नमः ॥ ॐ नमो गुह्याद् गुह्याय गोप्त्रे गोपतये तुभ्यं नमः ॥\\[6pt]
सर्वविद्याधिपतये ज्योतिरूपाय परमेश्वर-पराय नमः । अचेतनाचेतन व्योमिन् व्यापिन् अरूपिन् प्रथम-प्रथम तेजस्-तेजज्योतिरूप अग्नि अधूम अभस्म अनादि अनादेनादे नाना नाना धूधू धूधू ॐ भूः ॐ भुवः ॐ सुवः ॐ महः ॐ जनः ॐ तपः ॐ सत्यं अनेकानेक वामा वामा विभवे अव्यय अव्ययशान्ते शासन शमन अतीत धामन् अनिधन निधनोद्भव शिव शर्व सर्व सदाशिव । योगाधिपतये नमः । शर्व मुञ्च प्रथम तेजः भवः भावनाभूत भव भवोद्भव शक्त्या-शक्ति विद्याविद्य पुरुष अजात देश ॐ क्लीं श्लीं ॐ ह्लीं ग्लीं ॐ फट् फट् ॐ हुं हुं ॐ द्रीं द्रीं ह्रीं ह्रीं ह्रीं ह्रूं ह्रूं ह्रूं $\times$ ३ नमः॥\\ 
स्वस्ति स्वाहा स्वधा वषट् वौषट् हुं फट् अनन्तानन्द स्वज्वलित स्वाहा। हेमगर्भधराय स्वाहा सर्वभूत सुखप्रद सर्वसान्निद्ध्यकर ब्रह्म-विष्णु-रुद्र-पर अनार्चित अस्तुतास्तुत पूर्वस्थित पूर्वार्ध साक्षि साक्षि तुरु तुरु पतङ्ग पतङ्ग ज्ञान ज्ञान शब्द शब्द सूक्ष्म स्थूल शिव शर्व सदाशिव नमः ॥\\
ॐ आं ईं ऊं व्योमव्यापिने ॐ यां रां लां वां गां षां सां हां लां क्षां क्षीं क्षूं क्षैं क्षौं क्षः प्रशान्तपतये नमः । जगत्प्रभवे नमः । ॐ स्वाहान्तपतये नमः । तत्वात्मने नमः । शर्वात्मने नमः । अमृतात्मने नमः । रुद्राणां पतये नमः ॥\\
सर्वात्मने पराय परमेश्वराय योगाय योगसम्भवाय करकर कुरुकुरु सद्य भव भव भवोद्भव सर्वदेव सर्वकार्यप्रमथन सदाशिव प्रसन्न नमोऽस्तु ते स्वाहा ॥\\
ॐ सदाशिव नमः । ॐ मं शिवस्य हृदये ज्वालानि स्वाहा । ॐ शिवात्मक महातेजः सर्वज्ञश्चापराजितम् । आवर्तय महाघोर पिङ्गलं कवच-प्रभम् । आयाहि पिङ्गलं महाकवचं शिवाज्ञया मम हृदये बन्ध बन्ध पूर्व पूर्व भक्ति-सूक्ष्म वज्रधर वज्राशनि वज्रशरीर इदं शरीरमनुप्रविश्य सर्वदुष्टान् स्तम्भय स्तम्भय हुं फट् स्वाहा ॥\\
ॐ प्रस्फुर प्रस्फुर घोर-घोरतर-तनुरूप चट चट प्रचट कह कह वम वम घातय घातय हुं फट् ॐ जूं सः ॥\\
ॐ तन्महेशाय विद्महे महादेवाय धीमहि | तन्नः शिवः प्रचोदयात् ॥\\
ॐ गणाम्बिकायै विद्महे महातपायै धीमहि | तन्नो गौरी प्रचोदयात् ॥\\
ॐ निधनपतान्तिकाय स्वाहा । जलाय नमो जललिङ्गाय ते नमः । रुद्राय नमो रुद्रलिङ्गाय ते नमः । भवाय नमो भवलिङ्गाय ते नमः । शर्वाय नमो शर्वलिङ्गाय ते नमः । ऊर्ध्वाय नमो ऊर्ध्वलिङ्गाय ते नमः । शिवाय नमः शिवलिङ्गाय ते नमः ।

\end{document}