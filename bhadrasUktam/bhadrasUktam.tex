\documentclass[12pt]{article}
\usepackage{fontspec}
\usepackage[english]{babel} 
\newfontfamily\skt[Script=Devanagari]{Siddhanta}
\usepackage{parskip}
\usepackage[dvipsnames]{xcolor}
\usepackage{graphicx}
\graphicspath{ {images/} }
\usepackage{float}
\usepackage{wrapfig}
\usepackage{array}
\usepackage[margin=1.3in]{geometry}
\title{\textbf{ {\skt ॥ भद्र-सूक्तम् ॥ }}}
\author{{\skt कौत्सः }}
\date{}

\begin{document}
\maketitle
{\color{NavyBlue}{\large{\skt
भद्रो नो अग्निः सुहवो विभावसुर् भद्र इन्द्रः पुरुहूतः पुरुष्टुतः । \\
भद्रः सूर्य उरुचक्षा उरुव्यचा भद्रश् चन्द्रमाः समिथेषु जागृविः ॥१॥ \\

भद्रः प्रजा अजनयन्नः प्रजापतिर् भद्रः सोमः पावमानो वृषा हरिः । \\
भद्रस् त्वष्टा विदधद् रूपाण्य् अद्भुतो भद्रो नो धाता वरिवस्यतु प्रजाः ॥२॥ \\

भद्रस् तार्क्ष्यः सुप्रजस्त्वाय नो महाँ अरिष्टनेमिः पृतना युधा जयन् ।\\
भद्रो वायुर् मातरिश्वा नियुत्पतिर् वेनो गयस्फान उशन् सदा ऽस्तु नः ॥३॥\\

भद्रो मित्रो वरुणो रुद्र इद् वृधा भद्रो ऽहिर्बुध्न्यो भुवनस्य रक्षिता ।\\
भद्रो नो वास्तोष्पतिर् अस्त्व् अमीवहा भद्रः क्षेत्रस्य पतिर् विचर्षणिः ॥४॥\\

भद्रो विभुर् विश्वकर्मा बृहस्पतिर् भद्रो द्विषस्तपनो ब्रह्मणस्पतिः ।\\
भद्रः सुपर्णो अरुणो मरुत्-सखा भद्रो नो वातो अभिवातु भेषजी ॥५॥\\

भद्रो दधिक्रा वृषभः कनिक्रदद् भद्रः पर्जन्यो बहुधा विराजति ।\\
भद्रा सरस्वाँ उत नः सरस्वती भद्रो वशी भद्र इन्द्रः पुरूरवः ॥६॥\\

भद्रो नः पूषा सविता यमो भगो भद्रो ऽग्रज एकपाद् अर्यमा मनुः ।\\
भद्रो विष्णुर् उरुगायो वृषाहरिर् भद्रो विवस्वाँ अभिवातु नस्त्मना ॥७॥\\

भद्रा गायत्री ककुब् उष्णिहा विराड् भद्रानुष्टुप् बृहती पङ्क्तिर् अस्तु नः ।\\
भद्रा नस् त्रिष्टुब् जगती पुरुप्रिया भद्रातिच्छान्दा बहुधा विभूवरी ॥८॥\\

भद्रा नो राकानुमतिः कुहूः सुहृद् भद्रा सिनीवाल्य् अदितिर् मही ध्रुवा ।\\
भद्रा नो द्यौर् अन्तरिक्षं मयस्करं भद्रो ऽश्वो दक्षस्तनयाय नस् तुजे ॥९॥\\

भद्रो नः प्राणः सुमनाः सुवागसद् भद्रो अपानः सतनुः सहात्मना ।\\
भद्रं चक्षुर् भद्रम् इच्छोत्रम् अस्तु नो भद्रं न आयुः शरदो असच्छताम् ॥१०॥\\

भद्रेन्द्राग्नी नो भवताम् ऋतावृधा भद्रा नो मित्रावरुणा धृतव्रता ।\\
भद्राश्विना नो भवतां नवेदसा भद्रा द्यावा-पृथिवी विश्व-शंभुवा ॥१२॥\\

भद्रा न इन्द्रावरुणा रिशादसा भद्रा न इन्द्रा भवतां बृहस्पती ।\\
भद्रेन्द्राविष्णू सवनेषु यावृधा भद्रेन्द्रासोमा युधि दस्यु-हन्तमा ॥१३॥\\

भद्राग्नाविष्णू विदधस्य प्रसाधना भद्रा नो ऽग्नीन्द्रा वृषभा-दिवस्पती ।\\
भद्रा नो अग्नीवरुणा प्रचेतसा भद्राग्नीषोमा भवता नवेदसा ॥१४॥\\

भद्रा सूर्या-चन्द्रमसा कविक्रतू भद्रा सोमा भवतां पूषणा नः ।\\
भद्रेन्द्रावायू पृतनासु-सासही भद्रा सूर्याग्नी अजिता धनञ्जया ॥१५॥\\

भद्रा नः सन्तु वसवो वसुप्रजा भद्रा रुद्रा वृत्रहणा पुरन्धरा ।\\
भद्रा आदित्याः सुपसः सुनीतयो भद्रा राजानो मरुतो विरप्सिनः ॥१६॥\\

भद्रा न ऊमा सुहवाः शतश्रियो विश्वेदेवा मनवश् चर्षणीधृतः ।\\
भद्राः साध्या अभिभवः सूरचक्षसो भद्रा नः सन्त्व् ऋभवो रत्न-धातमाः ॥१७॥\\

भद्राः सर्वे वाजिनो वाजसातयो भद्रा ऋषयः पितरो गभस्तयः ।\\
भद्रा भृगवो ऽङ्गिरसः सुदानवो भद्रा गन्धर्वाप्सरसः सुदंशसः ॥१८॥\\

भद्रा आपः शुचयो विश्वभृत्तमा भद्राः शिवा यक्ष्मनुदो न ओषधीः ।\\
भद्रा गावः सुरभयो वयोवृधो भद्रा योषा उशतीर् देवपत्न्यः ॥१९॥\\

भद्राणि सामानि सदा भवन्तु नो भद्रा अथर्वाण ऋचो यजूंषि नः ।\\
भद्रा नक्षत्राणि शिवानि विश्वा भद्रा आशा अह्रुताः सन्तु नो हृदि ॥२०॥\\

संवत्सरा न ऋतवो मयोभुवो यो वा आयुवाः सुसराण्य् उत क्षपाः ।\\
मुहूर्ताः काष्टाः प्रदिशो दिशश् च सदा भद्रा सन्तु द्विपदे शं चतुष्पदे ॥२१॥\\

भद्रं पश्येम प्रचरेम भद्रं भद्रं वदेम शृणुयाम भद्रम् ।\\
तन्नो मित्रो वरुणो मा महन्ताम् अदितिः सिन्धुः पृथिवी उत द्यौः ॥२२॥\\
}}}\\

\end{document}