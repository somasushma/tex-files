\documentclass[12pt]{article}
\usepackage{geometry}
\geometry{margin=1in}
\usepackage{fontspec}
\usepackage[english]{babel} 
\newfontfamily\skt[Script=Devanagari]{Siddhanta}
\newfontfamily\sktad[Script=Devanagari]{AdobeDevanagari-Regular}
\newfontfamily\sktsh[Script=Devanagari]{Shobhika}
\usepackage[dvipsnames]{xcolor}
\usepackage{parskip}
\usepackage{graphicx}
\graphicspath{ {images/} }
\begin{document}
{\color{MidnightBlue}{\normalsize{\sktsh 
नारद ऋषिः । विराट् छन्दः । श्रीमद् धयग्रीवः परम-पुरुषो नारायणो देवता ॥\\

नमस् ते देवदेव [१] निष्क्रिय [२] निर्गुण [३] लोकसाक्षिन् [४] क्षेत्रज्ञ [५] अनन्त [६] पुरुष [७] महापुरुष [८] त्रिगुण [९] प्रधान [१०] अमृत [११] व्योम [१२] सनातन [१३] सदसद्व्यक्ताव्यक्त [१४] ऋतधामन् [१५] पूर्वादिदेव [१६] वसुप्रद [१७] प्रजापते [१८] सुप्रजापते [१९] वनस्पते [२०] महाप्रजापते [२१] ऊर्जस्पते [२२] वाचस्पते [२३] मनस्पते [२४] जगत्पते [२५] दिवस्पते [२६] मरुत्पते [२७] सलिलपते [२८] पृथिवीपते [२९] दिक्पते [३०] पूर्वनिवास [३१] ब्रह्मपुरोहित [३२] ब्रह्मकायिक [३३] महाकायिक [३४] महाराजिक [३५] चतुर्महाराजिक [३६] आभासुर [३७] महाभासुर [३८] सप्तमहाभासुर [३९] याम्य [४०] महायाम्य [४१] संज्ञासंज्ञ [४२] तुषित [४३] महातुषित [४४] प्रतर्दन [४५] परिनिर्मित [४६] वशवर्तिन् [४७] अपरिनिर्मित [४८] यज्ञ [४९] महायज्ञ [५०] यज्ञसंभव [५१] यज्ञयोने [५२] यज्ञगर्भ [५३] यज्ञहृदय [५४] यज्ञस्तुत [५५] यज्ञभागहर [५६] पञ्चयज्ञधर [५७] पञ्चकालकर्तृगते [५८] पञ्चरात्रिक [५९] वैकुण्ठ [६०] अपराजित [६१] मानसिक [६२] परमस्वामिन् [६३] सुस्नात [६४] हंस [६५] परमहंस [६६] परमयाज्ञिक [६७] सांख्ययोग [६८] अमृतेशय [६९] हिरण्येशय [७०] वेदेशय [७१] कुशेशय [७२] ब्रह्मेशय [७३] पद्मेशय [७४] विश्वेश्वर [७५] त्वं जगदन्वयः [७६] त्वं जगत्प्रकृतिः [७७] तवाग्निर् आस्यम् [७८] वडवामुखो ऽग्निः [७९] त्वम् आहुतिः [८०] त्वं सारथिः [८१] त्वं वषट्कारः [८२] त्वम् ॐकारः [८३] त्वं मनः [८४] त्वं चन्द्रमाः [८५] त्वं चक्षुर् आद्यम् [८६] त्वं सूर्यः [८७] त्वं दिशां गजः [८८] दिग्भानो [८९] हयशिरः [९०]  प्रथमत्रिसौपर्ण [९१] पञ्चाग्ने [९२] त्रिणाचिकेत [९३] षडङ्गविधान [९४] प्राग्ज्योतिष [९५] ज्येष्ठसामग [९६] सामिकव्रतधर [९७] अथर्वशिरः [९८] पञ्चमहाकल्प [९९] फेनपाचार्य [१००] वालखिल्य [१०१] वैखानस [१०२] अभग्नयोग [१०३] अभग्नपरिसंख्यान [१०४] युगादे [१०५] युगमध्य [१०६] युगनिधन [१०७] आखण्डल [१०८] प्राचीनगर्भ [१०९] कौशिक [११०]  पुरुष्टुत [१११] पुरुहूत [११२] विश्वरूप [११३] अनन्तगते [११४] अनन्तभोग [११५] अनन्त [११६] अनादे [११७] अमध्य [११८] अव्यक्तमध्य [११९] अव्यक्तनिधन [१२०] व्रतावास [१२१] समुद्राधिवास [१२२] यशोवास [१२३] तपोवास [१२४] लक्ष्म्यावास [१२५] विद्यावास [१२६] कीर्त्यावास [१२७] श्रीवास [१२८] सर्वावास [१२९] वासुदेव [१३०] सर्वच्छन्दक [१३१] हरिहय [१३२] हरिमेध [१३३] महायज्ञभागहर [१३४] वरप्रद [१३५] यम-नियम-महानियम-कृच्छ्रातिकृच्छ्र-महाकृच्छ्र-सर्वकृच्छ्र-नियमधर [१३६] निवृत्तधर्मप्रवचनगते [१३७] प्रवृत्तवेदक्रिय [१३८] अज [१३९] सर्वगते [१४०]  सर्वदर्शिन् [१४१] अग्राह्य [१४२] अचल [१४३] महाविभूते [१४४] माहात्म्यशरीर [१४५] पवित्र [१४६] महापवित्र [१४७] हिरण्मय [१४८] बृहत् [१४९] अप्रतर्क्य [१५०]  अविज्ञेय [१५१] ब्रह्माग्र्य [१५२] प्रजासर्गकर [१५३] प्रजानिधनकर [१५४] महामायाधर [१५५] चित्रशिखण्डिन् [१५६] वरप्रद [१५७] पुरोडाशभागहर [१५८] गताध्वन् [१५९] छिन्नतृष्ण [१६०]  छिन्नसंशय [१६१] सर्वतोनिवृत्त [१६२] ब्राह्मणरूप [१६३] ब्राह्मणप्रिय [१६४] विश्वमूर्ते [१६५] महामूर्ते [१६६] बान्धव [१६७] भक्तवत्सल [१६८] ब्रह्मण्यदेव [१६९] भक्तो ऽहं त्वां दिदृक्षुः [१७०] एकान्तदर्शनाय नमो नमः [१७१]
}}}
\end{document}