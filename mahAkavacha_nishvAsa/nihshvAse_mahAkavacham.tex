\documentclass[12pt]{article}
\usepackage[margin=1.3in]{geometry}
\usepackage{fontspec}
\usepackage[english]{babel} 
\newfontfamily\skt[Script=Devanagari]{Adobe Devanagari}
\newfontfamily\sktr[Script=Devanagari]{Adishila}
\usepackage{parskip}
\usepackage{float}
\usepackage{wrapfig}
\usepackage{array}
\usepackage{amsmath}
\usepackage{amssymb}
\usepackage{array}
\usepackage[dvipsnames]{xcolor}
\usepackage{graphicx}
\graphicspath{ {images/} }
\usepackage{tikz}
\usepackage{tikz-network}
\usetikzlibrary{arrows, arrows.meta, patterns, shapes.geometric, shapes.misc, graphs, mindmap, calc}
\title{\textbf{ {\color{Sepia}{\skt ॥ निश्वासकारिके महाकवचम् ॥ }}}}
\author{}
\date{}
\begin{document}
	\maketitle
\begin{center}
	\tikzset{
		petal3/.pic={
			\draw[fill=#1] (0,0) .. controls (0,3) and (1.5,0) .. (2,0) .. controls (1.5,0) and (0,-3) .. (0,0) -- cycle;
		}
	}
	\begin{tikzpicture}
	%petals
		\foreach \x in {0,45,90, 135,180, 225,270, 315}{
		\draw[Fuchsia, fill=Fuchsia!30!White, line width=.8] (\x:2.155) pic[scale=0.74, rotate=\x] {petal3};
	}
 %circles 
	\filldraw[line width=1, Blue] (0,0) circle(2);
	\draw[line width=1, Red] (0,0) circle(2.05);
	\draw[line width=1, Red] (0,0) circle(2.1);
	\draw[line width=1, Red] (0,0) circle(2.15);
%triangles
	\draw[line width=1, Goldenrod] (270:2) -- (30:2) -- (150:2) -- cycle;
  \draw[line width=1, Goldenrod] (0,1) -- (1.4012585384, -1.4270509831) -- (-1.401259, -1.427051) -- cycle;
  \draw[line width=1, Goldenrod] (-0.866025403784891, -0.5) -- (1.93649167310365, -0.5) -- (0.535233134659599, 1.92705098312447) -- cycle;
  \draw[line width=1, Goldenrod] (-1.93649167310365, -0.5) -- (0.866025403784891, -0.5) -- (-0.535233134659599, 1.92705098312447) -- cycle;
  \node[Goldenrod] (s1) at (0,0.1) {\large{\sktr{ह्सौं}}};
 %doors
\draw[Periwinkle, rounded corners=3pt, line width=1.5, double] (.5,-4.5) -- (.5, -4) -- (4, -4) -- (4,-.5) -- (4.5, -.5);
\foreach \x / \y in {-1/1, 1/-1, -1/-1}{
	\begin{scope}[xscale=\x, yscale=\y]
	\draw[Periwinkle, rounded corners=3pt, line width=1.5, double] (.5,-4.5) -- (.5, -4) -- (4, -4) -- (4,-.5) -- (4.5, -.5);
	\end{scope}}
\end{tikzpicture}\\	

\end{center}
{\color{NavyBlue}{\large{\sktr

शिवाङ्गाः – \\[5pt]
ॐ सर्वव्यापिने नमः । ॐ व्योमरूपाय नमः । ओम् अनन्ताय नमः । ओम् अनाथाय नमः । ओम् अनाश्रिताय नमः ॥\\[5pt]
विद्याङ्गा – \\[5pt]
ॐ ध्रुवाय नमः । ॐ शाश्वताय नमः । ॐ योगपीठसंस्थिताय नमः । ॐ नित्ययोगिने नमः । ॐ ध्यानाहाराय नमः ॥\\[5pt]
पदब्रह्मा – \\[5pt]
ओम् उं नमः । ॐ व्योमव्यापिने नमः । ॐ व्योमरूपाय नमः । ॐ सर्वव्यापिने नमः । ॐ शिवाय नमः । ओम् अनन्ताय नमः । ओम् अनाथाय नमः । ओम् अनाश्रिताय नमः । ॐ ध्रुवाय नमः । \\[5pt]
शिवगर्भः – \\[5pt]
ॐ शाश्वताय नमः । ॐ योगपीठसंस्थिताय नमः । ॐ नित्ययोगिने नमः । ॐ ध्यानाहाराय नमः । ओम् ओं नमः । ॐ शिवाय नमः । ॐ सर्वप्रभवे नमः । ॐ शिवाय नमः । ओम् ईशानमूर्धाय नमः । ॐ तत्पुरुषवक्त्राय नमः ॥ \\[5pt]
विद्येशावराः – \\[5pt]
ॐ अघोर-हृदयाय नमः । ॐ वामदेव गुह्याय नमः । ॐ सद्योजातमूर्तये नमः । ॐ ओं नमो नमः । ॐ गुह्यातिगुह्याय नमः । ॐ गोप्त्रे नमः । ॐ अनिधननिधनाय नमः । ॐ निधनाय नमः ॥ \\[5pt]
तृतीयावरणम् – \\[5pt]
ॐ सर्वविद्याधिपाय नमः । ॐ ज्योतीरूपाय नमः । ओम् परमेश्वरवराय नमः । ॐ चेतन नमः । ओम् अचेतन नमः । ॐ व्योमिन् व्योमिन् नमः । ॐ व्यापिन् व्यापिन् नमः । ॐ रूपिन् रूपिन् नमः । ॐ प्रथम प्रथम नमः । ॐ तेजस् तेज नमः ॥\\[5pt] 
गणेशावरणम् – \\[5pt] 
ॐ ज्योतिर्-ज्योतिन् नमः । ओम् अरूप नमः । ओम् अग्नि नमः । ॐ मधु नमः । ॐ भस्म नमः । ओम् अनादि नमः । ॐ नाना नमः । ॐ धू धू धू धू नमः । ॐ भूर् नमः ॥ \\[5pt]
आग्नेयावरणम् – \\[5pt]
ॐ भुवर् नमः । ॐ स्वर् नमः । ओम् अनिधन नमः । ॐ निधनोद्भव नमः । ॐ शिव नमः । ॐ सर्व नमः । ॐ परमात्मन् नमः । ॐ महेश्वर नमः ॥ \\[5pt]
लोकपालानामावरणम् – \\[5pt]
ॐ महादेव नमः । ॐ सद्भावेश्वर नमः । ॐ महातेज नमः । ॐ योगाधिपते नमः । ॐ मुञ्च मुञ्च मुञ्च नमः । ॐ प्रथम प्रथम नमः । ॐ शर्व शर्व नमः । ॐ भव भव नमः । ॐ भवोद्भव नमः ॥ \\[5pt]
महेन्द्रावरणम् – \\[5pt] 
ॐ सर्वभूतसुखप्रद नमः । ॐ सर्व सान्निध्यकर नमः । ॐ ब्रह्मा-विष्णु-रुद्र-पर नमः । ओम् अनर्चितानर्चित नमः । ॐ संस्कृतासंस्कृत नमः । ॐ पूर्वस्थित पूर्वस्थित नमः । ॐ साक्षि साक्षि नमः । ॐ तुरु तुरु नमः । ॐ पतङ्ग पतङ्ग नमः । ॐ पिङ्गपिङ्ग नमः ॥\\[5pt]
लोकपालास्त्रावरणं – \\[5pt]
ॐ शब्द-शब्द नमः । ॐ सूक्ष्म-सूक्ष्म नमः । ॐ शिव नमः । ॐ शर्व नमः । ॐ सर्व नमः । ओम् ॐ नमो नमः । ओम् ॐ शिवाय नमो नमः । ओम् ॐ नमः ॥\\
ओम् ऊं नमः । ॐ यं नमः । ॐ लं नमः । ओम् अं नमः । ॐ क्षं नमः । ॐ रं नमः । ॐ हं नमः । ॐ प्रसन्नमूर्तये नमः । ॐ सर्वात्मने पराय परमेश्वराय योगाधिपाय योगसंभवाय कर कर सद्यः भवोद्भव वामदेव सर्वकार्यप्रशमन मं सदाशिव प्रसन्न नमो नमः ॥\\[5pt]
सर्वात्म-देह-हृदयम् – \\[5pt]
ॐ सुशिव नमः । ॐ सुशिव देवशिरः । ॐ शिव-हृदय ज्वालिनी नमः । ज्वालिनी शिखा शिवात्मकं महातेजः । सर्वज्ञम् प्रभुम् अव्ययम् ॥ आवर्तयेन् महाघोरं कवचं पिङ्गलं शुभम् ॥\\[5pt] 
कवच हृदयम् – पिङ्गलो नाम यः कवचः \\[5pt]
आयाहि पिङ्गल महारव च शिवाज्ञया हृदयं बन्ध बन्ध प्रज्वल प्रज्वल व्याघूर्ण भक्ति-सुक्ष्मः । वज्रधर वज्रपाश वज्रशक्तिश्चन्द्र शरीर मम हृदयम् अनुप्रविश्य सर्वदुष्टान् स्तंभय हुं फण् नमः । ॐकार बन्धं हुं फट् ॥\\[5pt] 
अघोरास्त्रम् – अघोरास्त्रं हृदयम् ॥ \\[5pt] 
ॐ प्रस्फुर प्रस्फुर घोर घोरतर तनुरूप चट चट प्रचट प्रचट कह कह वं वं घातय घातय हुं फट् ॥ ॐ ॐ अघोरशिवाय हुं फट् ॥  \\[5pt]
व्योमव्यापिन् – परम् महानाथ शिवभट्टारकः ॥ \\[5pt] 
ॐ व्योमव्यापिने व्योमरूपाय सर्वव्यापिने शिवाय अनन्ताय अनाथाय अनाश्रिताय ध्रुवाय शाश्वताय योगपीठसंस्थिताय नित्ययोगिने ध्यानाहाराय ।ॐ नमः शिवाय सर्वप्रभवे शिवाय ईशान-मूर्धाय तत्पुरुष-वक्त्राय अघोर-हृदयाय वामदेव-गुह्याय सद्योजात-मूर्तये ॐ नमो नमः ॥\\[5pt]
गुह्यातिगुह्याय गोप्त्रे अनिधनाय सर्वयोगाधिकृताय सर्वविद्याधिपतये ज्योतिरूपाय परमेश्वर-पराय अचेतनाचेतन व्योमिन् व्योमिन् व्यापिन् व्यापिन् अरूपिन्नरूपिन् प्रथम प्रथमस् तेजस्तेजः ज्योतिर्ज्योतिः अरूप अनग्नि अधूम अभस्म अनादि नाना नाना धू धू धू धू ॐ भूः ॐ भुवः ॐ सुवः । अनिधन-निधन निधनोद्भव शिव-सर्व-परमात्मन् महेश्वर महादेव सद्भावेश्वर महातेज योगाधिपते मुञ्च मुञ्च प्रथम प्रथम शर्व शर्व भव भव भवोद्भव सर्व-भूत-सुखप्रद सर्वसान्निध्यकर ब्रह्म-विष्णु-रुद्र-पर अनर्चितानर्चित पर असंस्कृता संस्कृत पूर्वस्थित पूर्वस्थित साक्षि साक्षि तुरु तुरु पतङ्ग पतङ्ग पिङ्ग पिङ्ग ज्ञान ज्ञान शब्द शब्द सूक्ष्म सूक्ष्म शिव सर्व सर्वद । ॐ नमो नम । ॐ नमः । ॐ शिवाय नमो नमः । \\[5pt] 
पञ्च-ब्रह्म-मन्त्राणि – \\[5pt] 
ॐ सद्योजातं प्रपद्यामि सद्योजाताय वै नमः । भवे ̍नादिभवे भजस्व मां भवोद्भवाय नमः ॥\\[5pt] 
वामदेवाय नमो ज्येष्ठाय नमो रुद्राय नमः कालाय नमः कलविकरणाय नमो बलविकरणाय नमो बलप्रमथनाय नमः सर्वभूतदमनाय नमो मनोन्मनाय नमः ॥\\[5pt] 
ॐ अघोरेभ्योऽथ घोरेभ्यो घोरघोरतरेभ्यः । सर्वतः सर्वशर्वेभ्यो नमस्तेऽस्तु रुद्ररूपेभ्यः ॥\\[5pt] 
ॐ तत्पुरुषाय विद्महे महादेवाय धीमहि । तन्नो रुद्रः प्रचोदयात् ॥\\[5pt] 
ॐ ईशानः सर्वभूतानाम् ईश्वरः सर्वभूतानां ब्रह्माधिपतिर् ब्रह्मणोऽधिपतिर् ब्रह्मा शिवो मेऽस्तु सदाशिवोम् ॥\\[5pt] 
ॐ तन् महेशाय विद्महे वाग्विशुद्धाय धीमहि । तन्नः शिवः प्रचोदयात् ॥\\[5pt] 
शिवात्मकम् इदं सर्वं शिवादेव प्रवर्तते ॥ शिवाय शिवगर्भाय शिवः सर्वः प्रचोदयात् ॥\\[5pt] 
ॐ आं ईं ऊं व्योमव्यापिने ॐ ॐ प्रशान्ताय नमः । ॐ प्रणवात्मने नमः । ॐ हुं शिवाय नमः । ॐ हुं नमः । ॐ फण् नमः । ॐ जूं नमः ॥\\[5pt] 
ॐ सद्योजातं प्रपद्यामि नमः । ॐ सद्योजाताय वै नमः । ॐ भवे नमः । ॐ अभवे नमः । ॐ अनादिभवे नमः । ॐ भजस्व मां नमः । ॐ भव नमः । ॐ उद्भवाय नमः । नमः ॐ वामदेवाय नमो नमः । ॐ ज्येष्ठाय नमो नमः । ॐ रुद्राय नमो नमः । ॐ कालाय नमो नमः । ॐ कल नमः । ॐ विकरणाय नमो नमः । ॐ बल नमः । ॐ विकरणाय नमो नमः । ॐ बल नमः । ॐ प्रमथनाय नमो नमः । ॐ सर्वभूतदमनाय नमो नमः । ॐ मनन नमः । ॐ उन्मनाय नमो नमः । ॐ अघोरेभ्यो नमः । ओम् अघोरेभ्यो नमः । ॐ घोर नमः । ॐ घोरतरेभ्यो नमः । ॐ सर्वतो नमः । ॐ सर्वशर्वेभ्यो नमः । ॐ नमस्ते अस्तु रुद्र नमः । ॐ रूपेभ्यो नमः । ॐ तत्पुरुषाय विद्महे नमः । ॐ महादेवाय धीमहे नमः । ॐ तन्नो रुद्र नमः । ॐ प्रचोदयान् नमः । ओम् ईशानः सर्वविद्यानां नमः । ॐ ईश्वरः सर्वभूतानां नमः । ॐ ब्रह्माधिपतिर् ब्रह्मणोधिपतिर् ब्रह्मा नमः । ॐ शिवोमेऽस्तु नमः । ॐ सदाशिवों नमः ॥
}}}
\end{document}