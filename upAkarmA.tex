\documentclass[12pt]{article}
 \usepackage{fontspec}
 \usepackage[english]{babel} 
 \newfontfamily\skt[Script=Devanagari]{Adobe Devanagari}
 \usepackage[dvipsnames]{xcolor}
 \usepackage{parskip}
\begin{document}
{\color{Black}{\Large{\skt
आचमनम् : सावित्र्या पादैस् त्रिवारं जलम् उच्चुलुम्पेत् ।
तत् सवितुर् वरेण्यं । भर्गो देवस्य धीमहि । धियो यो नः प्रचोदयात् ॥\\

प्राणायामः :\\
पूरकं कुर्यात्: ॐ भूः ।\\
कुंभकं कुर्यात्: ॐ भुवः । ओग्ं सुवः । ॐ महः । ॐ तपः । ओग्ं सत्यम् । ॐ तत् सवितुर् वरेण्यं । भर्गो देवस्य धीमहि । धियो यो नः प्रचोदयात् ।\\
रेचकं कुर्यात्: ॐ आपो ज्योती रसो ऽमृतं ब्रह्म भूर्-भुवस्-स्वरों ॥\\

ममोपात्त समस्त-दुरितक्षय-द्वारा सर्वा देवताः प्रीत्यर्थं शुभे शोबने मुहूर्ते आद्य ब्रह्मणो द्वितीय परार्धे श्वेत-वराह-कल्पे वैवस्वत-मन्वन्तरे अष्टविंशति-तमे कलि-युगे प्रथमे पादे क्रौञ्च-द्वीपे मध्यम-म्लेच्छ-वर्षे म्लेच्छ-खण्डे पोटोमाक-नदि-तटे मेरोर् दक्षिण-पार्श्वे अस्मिन् वर्तमाने शकाब्दे प्रभवादि-षष्ठि-संवत्सराणां-मध्ये <विलंबी>-नाम-संवत्सरे दक्षिणायने ग्रीष्म-ऋतौ श्रावण-मासे शुक्ल-पक्षे पौर्णमास्यां <रवि>-वासर-युक्तायां श्रवण-नक्षत्र-युक्तायां शुभ-योग-शुभ-करण-एवम्-गुण-विशेषेण विशिष्ठायाम् अस्यां पौर्णमास्यां शुभ-तिथौ उत्सर्जनाकरण-प्रायश्चित्तार्थम् अष्टोत्तर-<शत/सहस्र>-संख्यया कामो ऽकार्षीत् मन्युर् अकार्षीत् इति महामन्त्र जपं करिष्ये ।\\

कामो ऽकार्षीन् नमो नमः । कामो ऽकार्षीत् कामः करोति नाहं करोमि कामः कर्ता नाहं कर्ता कामः कारयिता नाहं कारयिता एष ते काम कामाय स्वाहा ॥\\

मन्युर् अकार्षीन् नमो नमः । मन्युर् अकार्षीन् मन्यु करोति नाहं करोमि मन्युः कर्ता नाहं कर्ता मन्युः कारयिता नाहं कारयिता एष ते मन्यो मन्यवे स्वाहा ॥\\

आचमनम्\\

प्राणायामः\\

ममोपात्त समस्त-दुरितक्षय-द्वारा सर्वा देवताः प्रीत्यर्थं शुभे शोबने मुहूर्ते आद्य ब्रह्मणो द्वितीय परार्धे श्वेत-वराह-कल्पे वैवस्वत-मन्वन्तरे अष्टविंशति-तमे कलि-युगे प्रथमे पादे क्रौञ्च-द्वीपे मध्यम-म्लेच्छ-वर्षे म्लेच्छ-खण्डे पोटोमाक-नदि-तटे मेरोर् दक्षिण-पार्श्वे अस्मिन् वर्तमाने शकाब्दे प्रभवादि-षष्ठि-संवत्सराणां-मध्ये < विलंबी>-नाम-संवत्सरे दक्षिणायने ग्रीष्म-ऋतौ श्रावण-मासे शुक्ल-पक्षे पौर्णमास्यां <रवि>-वासर-युक्तायां <श्रवण>-नक्षत्र-युक्तायां शुभ-योग-शुभ-करण-एवम्-गुण-विशेषेण विशिष्ठायाम् अस्यां <पौर्णमास्यां> शुभ-तिथौ श्रौत-स्मार्त-विहित-नित्यकार्मानुष्ठान-योग्यता-सिद्ध्य् अर्थं यज्ञोपवीत-धारणं करिष्ये ।\\

नुतनं यज्ञोपवीतं हस्ताभ्यां गृहीत्वा मन्त्रम् इदं वदेत् :\\

यज्ञोपवीत-धारण-मन्त्रस्य\\
ब्रह्मा ऋषिः । (मस्तकं स्पृशेत्)\\
त्रिष्टुभ् छन्दः । (उत्तरोष्ठं स्पृशेत्)\\
परमात्मा देवता । (वक्षः स्पृशेत्)\\
यज्ञोपवीत-धारणे विनियोगः ।\\

यज्ञोपवीतं परमं पवित्रं प्रजापतेर् यत् सहजं पुरस्तात् ।\\
आयुष्यम् अग्रयं प्रतिमुञ्च शुभ्रं यज्ञोपवीतं बलम् अस्तु तेजः ॥\\

आचमनम्\\

उपवीतं भिन्न-तन्तुं जीर्णं कश्मल-दूशितम् ।\\
विसृजामि पुनर् ब्रह्मन् वर्चो दीर्घायुर् अस्तु मे ॥\\
(पुरातनम् उपवीतम् विसृजेत्)\\

अथ काण्ड-ऋषि-तर्पणम् । (उपवीतं कण्ठी कृत्वा श्वेत-तिलांश् चोदकं च प्रसिञ्चेत्)\\

अथ कनिष्ठिकयोर् अधो भागात् प्रसिञ्चेत् :\\ 
प्रजापतिं काण्डर्षिं तर्पयामि ।\\
सोमं काण्डर्षिं तर्पयामि ।\\
अग्निं काण्डर्षिं तर्पयामि ।\\
विश्वान् देवान् काण्डर्षिं तर्पयामि ।\\
साग्ं-हितीर् देवता उपनिषदस् तर्पयामि ।\\
याज्ञीकीर् देवता उपनिषदस् तर्पयामि ।\\
वारुणीर् देवता उपनिषदस् तर्पयामि । \\

अथ करयोर् अधो भागात् प्रसिञ्चेत् :\\
ब्रह्माणग्ं स्वायंभुवं तर्पयामि ।\\

अथ अङ्गुलीनाम् उत्तरभागात् प्रसिञ्चेत् :\\
सदसस्पतिं तर्पयामि । ऋग्वेदं तर्पयामि ।यजुर्वेदं तर्पयामि ।सामवेदं तर्पयामि ।अथर्वणवेदं तर्पयामि । इतिहासं तर्पयामि ।पुराणं तर्पयामि ।कल्पं तर्पयामि ।
इन्द्रं तर्पयामि । विष्णुं तर्पयामि । वरुणं तर्पयामि । मित्रं तर्पयामि । अर्यमनं तर्पयामि । सवितारं तर्पयामि । बृहस्पतिं तर्पयामि । सोमं तर्पयामि । रुद्रं तर्पयामि । मरुतस् तर्पयामि । अश्विनौ तर्पयामि । सर्वा देवीस् तर्पयामि ।\\

आचमनम् । इति यजुष्-शाखाध्यायिनाम् उपाकर्म समाप्तम् ॥\\

अथ ऋक्-शाखाध्यायिनाम् तर्पण-विधिः :\\
(श्वेत-तिलांश् चोदकं च पात्रात् उद्धरण्या प्रत्येकेन मन्त्रेण प्रसिञ्चेत्)\\

अग्निं तर्पयामि । मरुतस् तृप्यन्तु । अग्निं तर्पयामि । वर्माणं तर्पयामि । अग्निं तर्पयामि । मित्रावरुणौ तृप्येतां । अग्निं तर्पयामि । इन्द्रासोमौ तृप्येतां । इन्द्रं तर्पयामि । अग्निमरुतौ तृप्येतां । पवमान-सोमौ तृप्येतां । सोमं तर्पयामि । अग्निं तर्पयामि । संज्ञानं तर्पयामि । विश्वेदेवास् तृप्यन्तां । देवतास् तृप्यन्तां ।\\

अग्निं तृप्यतु । विष्णुं तृप्यतु । प्रजापतिं तृप्यतु । ब्रह्माणं तृप्यतु ॥\\

वेदान् तृप्यन्तु ।\\

देवास् तृप्यन्ताम् । ऋष्यस् तृप्यन्ताम् । सर्वाणि छन्दाम्सि तृप्यन्ताम् । ॐकारस् तृप्यताम् । वषट्कारस् तृप्यताम् । व्याहृतयस् तृप्यन्ताम् । सावित्री तृप्यताम् । यज्ञास् तृप्यन्ताम् । द्यावा-पृथिवी तृप्येताम् । अन्तरिक्षं तृप्यताम् । अहो-रात्रानि तृप्यन्ताम् । संख्यास् तृप्यन्ताम् । सांख्यास् तृप्यन्ताम् । सिद्धास् तृप्यन्ताम् । साध्यास् तृप्यन्ताम् । समुद्रास् तृप्यन्ताम् । नद्यस् तृप्यन्ताम् । गावस् तृप्यन्ताम् । गिरयस् तृप्यन्ताम् । क्षेत्रोषधि-वनस्पति-गन्धर्वाप्सरसस् तृप्यन्ताम् । नागास् तृप्यन्ताम् । वयांसि तृप्यन्ताम् । विप्रास् तृप्यन्ताम् । यक्षास् तृप्यन्ताम् । भूतानि तृप्यन्ताम् ।\\

शतर्चिनस् तृप्यन्ताम् । मध्यमास् तृप्यन्ताम् । गृत्समदस् तृप्यताम् । विश्वामित्रस् तृप्यताम् । वाम्देवस् तृप्यताम् ।अत्रिस् तृप्यताम् । भरद्वाजस् तृप्यताम् ।वसिष्ठस् तृप्यताम् । प्रगाथास् तृप्यन्ताम् । पावमान्यस् तृप्यन्ताम् । क्षुद्र-सूक्तास् तृप्यन्ताम् । महा-सूक्तास् तृप्यन्ताम् । साम-गानानि तृप्यन्ताम् ॥\\

आचमनम् । इति ऋक्-शाखाध्यायिनाम् उपाकर्म समाप्तम् ॥\\

अथ अथर्वण-शाखाध्यायिनाम् तर्पण-विधिः :\\
(श्वेत-तिलांश् चोदकं च पात्रात् उद्धरण्या प्रत्येकेन मन्त्रेण प्रसिञ्चेत्)\\

आङ्गिरसानाम् आद्यैः पञ्चानुवाकैः स्वाहा । षष्ठाय स्वाहा । सप्तमाष्टमाभ्यां स्वाहा नीलनखेभ्यः स्वाहा । हरितेभ्यः स्वाहा ।क्षुद्रेभ्यः स्वाहा ।पर्यायिकेभ्यः स्वाहा । प्रथमेभ्यः शङ्खेभ्यः स्वाहा । द्वितीयेभ्यः शङ्खेभ्यः स्वाहा । तृतीयेभ्यः शङ्खेभ्यः स्वाहा । ऋषिभ्यः स्वाहा ।शिखिभ्यः स्वाहा । गणेभ्यः स्वाहा । महागणेभ्यः स्वाहा । सर्वेभ्योऽङ्गिरोभ्यो विदगणेभ्यः स्वाहा । पृथक्सहस्राभ्यां स्वाहा । ब्रह्मणे स्वाहा । \\

ब्रह्मज्येष्ठा सम्भृता विर्याणि ब्रह्माग्रे ज्येष्ठं दिवम् आ ततान ।\\
भूतानां ब्रह्मा प्रथमोत जज्ञे तेनार्हति ब्रह्मणा स्पर्धितुं कः ॥\\

आथर्वणानां चतुरृचेभ्यः स्वाहा । पञ्चर्चेभ्यः स्वाहा । षऌऋचेभ्यः स्वाहा । सप्तर्चेभ्यः स्वाहा । अष्टर्चेभ्यः स्वाहा । नवर्चेभ्यः स्वाहा । दशर्चेभ्यः स्वाहा । एकादशर्चेभ्यः स्वाहा । द्वादशर्चेभ्यः स्वाहा । त्रयोदशर्चेभ्यः स्वाहा । चतुर्दशर्चेभ्यः स्वाहा । पञ्चदशर्चेभ्यः स्वाहा । षोडशर्चेभ्यः स्वाहा । सप्तदशर्चेभ्यः स्वाहा । अष्टादशर्चेभ्यः स्वाहा । एकोनविंशतिः स्वाहा । विंशतिः स्वाहा । महत्काण्डाय स्वाहा । तृचेभ्यः स्वाहा । एकर्चेभ्यः स्वाहा । क्षुद्रेभ्यः स्वाहा । एकानृचेभ्यः स्वाहा । रोहितेभ्यः स्वाहा । सूर्याभ्यां स्वाहा । व्रात्याभ्यां स्वाहा । प्राजापत्याभ्यां स्वाहा । विषासह्यै स्वाहा । मङ्गलिकेभ्यः स्वाहा । ब्रह्मणे स्वाहा ।\\

ब्रह्मज्येष्ठा संभृता वीर्याणि ब्रह्माग्रे ज्येष्ठं दिवम् आ ततान ।\\
भूतानां ब्रह्मा प्रथमोत जज्ञे तेनार्हति ब्रह्मणा स्पर्धितुं कः ॥\\

 अथ सावित्री-महामन्त्र जपं ।\\

ममोपात्त समस्त-दुरितक्षय-द्वारा सर्वा देवताः प्रीत्यर्थं शुभे शोबने मुहूर्ते आद्य ब्रह्मणो द्वितीय परार्धे श्वेत-वराह-कल्पे वैवस्वत-मन्वन्तरे अष्टविंशति-तमे कलि-युगे प्रथमे पादे क्रौञ्च-द्वीपे मध्यम-म्लेच्छ-वर्षे म्लेच्छ-खण्डे पोटोमाक-नदि-तटे मेरोर् दक्षिण-पार्श्वे अस्मिन् वर्तमाने शकाब्दे प्रभवादि-षष्ठि-संवत्सराणां-मध्ये < विलंबी>-नाम-संवत्सरे दक्षिणायने ग्रीष्म-ऋतौ श्रावण-मासे <कृष्ण>-पक्षे प्रथम्यां < सोम>-वासर-युक्तायां <श्रविष्ठा>-नक्षत्र-युक्तायां शुभ-योग-शुभ-करण-एवम्-गुण-विशेषेण विशिष्ठायाम् अस्यां प्रथम्यां शुभ-तिथौ उत्सर्जनाकरण-प्रायश्चित्तार्थम् अष्टोत्तर-<शत/सहस्र>-संख्यया सावित्री महामन्त्र जपं करिष्ये ।\\

सावित्र्या\\
गाथिनो विश्वामित्रो ऋषिः । (मस्तकं स्पृशेत्)\\
निचृद् गायत्री छन्दः । (उत्तरोष्ठं स्पृशेत्)\\
सविता देवता । (वक्षः स्पृशेत्)\\

ॐ भूर्-भुवः-सुवः । तत् सवितुर् वरेण्यं ।\\
भर्गो देवस्य धीमहि । धियो यो नः प्रचोदयात् ॥\\
}}}
\end{document}