\documentclass[12pt]{article}
 \usepackage{fontspec}
 \usepackage[english]{babel} 
 \newfontfamily\skt[Script=Devanagari]{Siddhanta}
 \usepackage{parskip}
 \usepackage[dvipsnames]{xcolor}
 \usepackage{graphicx}
\graphicspath{ {images/} }
\usepackage{float}
\usepackage{wrapfig}
\usepackage{array}
\title{\textbf{ {\skt ॥ अर्थशास्त्रस्य प्रकीर्णानि नुतनानि परिशिष्टानि ॥
}}}
\author{{\skt भर्गवः
}}
\date{}

\begin{document}
\maketitle

\section{{\skt मङ्गलाचरणम्
}}
{\skt नमामि देवदेवेशम् इन्द्रं वज्रधरं प्रभुं । \\
आदित्यान् रुद्रान् वसून् वन्देऽहम् अश्विनौ सह । \\
हुवेऽहं सजोषसा इन्द्राणीं सरस्वतीं रोदसीं च सर्वाभिर् ग्नाभिर् युताः ।\\
यमश् च भृगवश् चाङ्गिरश् च ब्रह्म-क्षत्रं सह तृप्यन्तु ॥\\
}
\section{{ \skt मूलसूत्राणि
}}
{ \skt अयं निजः परो वेति जीवस्य+ अभिन्नं मूलम् ।\\
तथा हि जीवस्य मूले संग्रामं ।\\
संग्रामाद् अजायत जिवस्य प्रपञ्चः ।\\
संग्रामेण उद्भवन्ति जिवाणुनां नवा नवा उपायाः ।\\
अनाभिका हि परमाः शिक्षकाः । \\
उपलूतिका-लूतिका-वरोल्य्-अलि-पिपीलक-वम्र-गणा उत्तरम् ।\\
मत्स्य-वि-जरायुजास् तदनन्तरम् ।\\
जरायुजानां वानरोपनरा विशेषताः प्रकृतयः ।\\
}
\section{{\skt विषमत्व-सूत्राणि
}}
{\skt लोके समता नास्ति ।\\
समत्वस्य+अपेक्षा ऽसंशयं मूढत्वम् ।\\
जातौ विषमत्वाद् वर्णस्य व्युत्पत्तिः ।\\
समत्वाद् विषमत्वाच् च समाजे पृतना ।\\
वर्ण-भेदस्य+अतिकृत्या दोषः समत्वाद् अपि\\
पृतनाया नियन्त्रणं स्थिर-परिभाषिते वर्णे ।\\
वर्णाद् लोक-विजयं वरोल्यल्यादि सांसर्गिक-कीटानाम् ।\\
}
\section {{\skt शत्रु-सूत्राणि
}}
{ \skt एक-राक्षस-मतानि मानसिका रोगाः ।\\
उन्मादाः ।\\
प्रथमोन्मादः प्रेतोन्मादो राक्षसोन्मादश् च+उन्माद-प्रकाराः ।\\
रुधिरोन्मादश् चानियमवादश् च+अशासनवदश् चोन्मादस्य निगुढाः प्रकाराः ।\\
एकराक्षस-मतानि धर्मस्य स्वाभाविकाः प्रतिनिविष्टाः शत्रवः ।\\
इन्द्रो वा वृत्रो वा तद्वद् धर्मो वा+एकराक्षसमतं वा ।\\
जातौ क्लीबानां सम्लिङ्गकामिनां च+अतिरेकम् अनैसर्गिकम् ।\\
तेषां प्रमाणाधिकं प्रदर्शनं मैथुनिक-प्राणिषु प्रायेण रोगेण वा विमोहनेन वा ।\\
तेषां प्रमाणाधिकम् अवधानम् अशासनवादिनां समाज-भञ्जनाय युक्तिः ।\\
पुंश्चली-वृत्तेः संकीर्तनं राष्ट्रस्य नष्टस्य स्पष्टं कारणम् ।\\
उन्मादमूलं मूषनाम-दुष्टेन प्रोक्तम् ।\\
संसक्तौ बलं प्रथमोन्मत्तानाम् ।\\
विधर्मं प्रस्तृण्वन्ति ।\\
स्वार्थाय राष्ट्र-भेदिनः ।\\
प्रथमोन्मत्तेन प्रेतेन प्राभवत् प्रेतोन्मादः ।\\
प्रेतोन्मदिताः पाश्चात्या इति म्लेच्छाः ।\\
प्रेतोन्मादो वै म्लेच्छानां व्यक्तित्वस्य स्कम्भः ।\\
म्लेच्छ-प्रोत्साहितो विधर्मस् तस्य निगुढं रूपम् ।\\
म्लेच्छाः प्रेतोन्मादेन+अन्यान् धर्मान् रुजन्ति ।\\
राष्ट्राय प्रेतोन्मदितं कुर्वन्ति ।\\
सम्प्रत्य् अस्मै कार्याय सूक्ष्मोपायान् प्रयुङ्जते ।\\
Maculinea चित्रपतङ्गस्य Myrmica पिप्पीलकोपर्य् आक्रमणम् इव ।\\
किं तु परिवर्तितेभ्यो जनेभ्यो म्लेच्छ-समम्-पदवीं न कदापि ददति ।\\
पूर्वोन्मादाच् च राक्षस-ग्रहणाच् च मरुसंभवो महामदो राक्षसोन्मादं वा मरून्मादं वा+अख्यापयत् ।\\
तस्य चिह्नानि मरणं मारणं चैव दूषणम् एव च ।\\
बहु-प्रजावन्तो लम्पटा बलात्काररताः कलहप्रियाः ।\\
लिङ्गछेदाच् छिरश्छेदाच् च प्रसारयन्ति मरून्मत्ताः ।\\
मरून्मादो ऽन्तर्-बहिः शोणितमयः ।\\
अभ्यन्तरेण युद्धेन मरून्मादस्य+ओजो नवो नवो भवति ।\\
}
\section {{\skt जन-संग्राम-सूत्राणि
}}
{\skt आर्याः पार्शवस् तुखारा यवनो रोमाका लिथुवाः श्रवाः शूलपुरुषाः केल्टाश् च+ इत्य् आर्यादि-गणः ।\\
तेषु आर्याः पार्शवश् च मुक्त्वा बहुशो म्लेच्छ्य् अभवन् वा लयम् आप्नुवन् वा ।\\
पूर्वे काले मरकताः प्रसिद्धः संस्कृतिम् अकृण्वन् ।\\
प्रेतोन्मत्तैर् मरून्मत्तैर् नाशम् आप्नुवन् ।\\
भोटाश् चीना इरावन्तश् चीन-भृत्याश् चम्पा उषापुत्राश् च +इति प्राच्यादि-गणः ।\\
अजगर इव चीना गिरन्ति भोटान् ।\\
चीनानां गृद्धेः शत्रुता ।\\
उषापुत्रा विविक्ताश् च तीव्रशिक्षार्थिनश् च ।\\
तेन महाबलम् आप्नुवन् परं तु म्लेच्छ-युक्त्या पराजिता अभवन् ।\\
चम्पा महासमरे म्लेच्छान् अजयन् ।\\
म्लेच्छ-प्रयोगेण चीनभृत्या अर्धाः प्रेतोन्मादिता अभवन् ।\\
पूर्वं सुवर्ण्द्वीपीया धर्मसाधकाः ।\\
बलीद्वीपीयस्य व्यतिरिक्तं मरून्मत्ताश् च प्रेतोन्मत्ताश् च+अभवन् ।\\
क्रौञ्चद्वीपीयाः पृथग्विधाः ।\\
हिरण्यस्य व्यतिरिक्तं न जानन्त्य् अयः किं तु तेषु केचित् कगदम् अतक्ष्नुवन् ।\\
केचित् कन्दुकक्रीडाम् अकल्पयन् ।\\
म्लेच्छ-प्रहारेण म्लेच्छ-प्रसारित-रोगेण यान्ति निधनम् ।\\
}
\section{{\skt धर्माधर्म-स्पर्धा-सूत्राणि
}}
{ \skt नैसर्गिका धर्मा मानवानां नैसर्गिकावस्थया प्रासुवत ।\\
प्रायेण प्राकृत्यवरणेन ।\\
उन्मादा नैसर्गिक-धर्माणां विरोधं कुर्वन्ति च तेषां संपूर्णं लयं अभिलषन्ति ।\\
तस्मिन् उद्भवति नैसर्गिक-धर्म-साधकानां प्रति सर्वोन्मत्ताभिसंधिः ।\\
सर्वोन्मत्तानां निरृतिगतेर् नैसर्गिकधर्माणाम् अवस्थानम् ।\\
एष हि युद्धो जीवाय मृत्यवे वा ।\\
शक्तिर् अस्माकं गणीकरणे ।\\
स्वात्ययाद् धि उत्तरदायित्वम् ।\\
विश्वस्य लोकस्य बोधनाय परमार्थग्रहणाय गणीकरणं गुरुतमम् ।\\
उत्तरतर न्याय-प्रदायकः शक्ति लोके नास्ति । \\
अतः संपूर्णस्य बलस्य साधने हि राष्ट्रस्याभ्युदयः ।\\
गुह्य-समाजम् अवश्यम् ।\\
}
\section{{\skt निघण्ठुः
}}
\begin{tabular}{rl}
{\skt अनाभिकः
} & Bacteria \\
{\skt प्रनाभिकः 
} & Archaea \\
{\skt नाभिकः 
} & Eukaryotes \\
{\skt उपनरः 
} & Apes \\
{\skt विधर्मः 
} & Secularism \\
{\skt मरकतः 
} & Egyptian \\
{\skt इरावत् 
} & Burman \\
{\skt प्राकृत्य-वरणम्
} & Natural selection\\
\end{tabular}
\end{document}