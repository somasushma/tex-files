\documentclass[12pt]{article}
 \usepackage{fontspec}
 \usepackage[english]{babel} 
 \newfontfamily\skt[Script=Devanagari]{Adobe Devanagari}
 \usepackage[dvipsnames]{xcolor}
 \usepackage{parskip}
\usepackage[margin=1.4in]{geometry}
\usepackage{amsmath}
\usepackage{amssymb}
\begin{document}
{\color{MidnightBlue}{\Large{\skt  असुर-सुर-वृन्द-वन्दितम् अभिमत-वर-वितरणे निरतम् ।\\
 दर्शन-शताग्र्य-पूज्यं प्राण-तनुं गणपतिं वन्दे ॥}}}\\
I worship \textit{Gaṇeśa}, of the form of \textit{prāṇa}, worshiped by the multitude of \textit{asura-s} and \textit{deva-s}, engaged in giving favorable boons, who is worshiped at the start in the hundreds of traditions.\\

{\color{MidnightBlue}{\Large{\skt वर-वीर-योगिनी-गण-सिद्धा-वलि-पूजितांघ्रि-युगलम् ।\\
 अपहृत-विनयि-जनार्तिं वटुकम् अपानाभिधं वन्दे ॥}}}\\
I worship \textit{Vaṭuka}(the boy), whose two feet are worshiped by the lineages of \textit{vīra-s}, \textit{yoginī-s} and \textit{siddha-s}, who takes away the troubles of those bowing in obeisance and splits the lower plexus.\\

{\color{MidnightBlue}{\Large{\skt आत्मीय-विषय-भोगैर् इन्द्रिय-देव्याः सदा हृद् अम्भोजे ।\\ 
अभि-पूजयन्ति यं तं चिन्मयम् आनन्दभैरवं वन्दे ॥}}}\\
I worship that \textit{Ānandabhairava} of the form of consciousness whom the goddesses, who enjoy the first person experience via the field of sense organs, ever worship in the lotus of the heart.\\

{\color{MidnightBlue}{\Large{\skt यद्-धीबलेन विश्वं भक्तानां शिव-पथं भाति ।\\ 
तम् अहम अवधान-रूपं सद्गुरुम् अमलं सदा वन्दे ॥}}}\\
I ever worship the preceptor, in his pure meditation-form, who by the strength of his intellect enlightens all the devotees on the path of \textit{Śiva}.\\

{\color{MidnightBlue}{\Large{\skt उदयावभास-चर्वण-लीलं विश्वस्य या करोत्य् अनिशम् ।\\ 
आनन्दभैरवीं तां विमर्श-रूपाम् अहम् वन्दे ॥}}}\\
I worship that \textit{Ānandabhairavī} of the form of knowledge, who constantly performs the play of the emergence, expansion and dissolution of the universe.\\

{\color{MidnightBlue}{\Large{\skt अर्चयति भैरवं या निश्चय-कुसुमैः सुरेश-पत्रस्था ।\\ 
प्रणमामि बुद्धिरूपां ब्रह्माणीं ताम् अहम् सततम् ॥}}}\\
I ever salute that \textit{Brāhmāṇī} of the form of intellect, who worships the \textit{Bhairava} with flowers of cognition in the petal of the lord of the gods (\textit{Indra}).\\
\newpage
{\color{MidnightBlue}{\Large{\skt कुरुते भैरव-पूजाम् अनल दलस्थाभिमान-कुसुमैर् या ।\\
 नित्यम् अहंकृति रूपां वन्दे तां शांभवीम् अम्बाम् ॥}}}\\
I ever worship that mother \textit{Śāmbhavī} of the form of ego, who performs \textit{Bhairava}-worship with flowers of I-ness in the petal of \textit{Agni}.\\

{\color{MidnightBlue}{\Large{\skt विदधाति भैरवार्चां दक्षिण-दलगा विकल्प-कुसुमैर् या ।\\ 
नित्यं मनः स्वरूपां कौमारीं ताम् अहं वन्दे ॥}}}\\
I ever worship that \textit{Kaumārī} of the form of the mind, who offers \textit{Bhairava}-worship with flowers of diverse thoughts in the southern petal (i.e. of \textit{Yama}).\\

{\color{MidnightBlue}{\Large{\skt नैरृत-दलगा भैरवम् अर्चयते शब्द-कुसुमैर् या ।\\ 
प्रणमामि श्रुति-रूपां नित्यं तां वैष्णवीं शक्तिं ॥}}}\\
I ever salute that \textit{śakti Vaiṣṇavī} of the form of hearing, who worships the \textit{Bhairava} with flowers of sound in the petal of \textit{Nirṛti}.\\

{\color{MidnightBlue}{\Large{\skt पश्चिम-दिग्-दल-संस्था हृदय-हरैः स्पर्श-कुसुमैर् या ।\\
 तोषयति भैरवं तां त्वग्-रूप-धरां नमामि वाराहीम् ॥}}}\\
I salute that \textit{Vārāhī} bearing the form of skin, who pleases the \textit{Bhairava} in the heart-lotus with flowers of touch in the western petal.\\

{\color{MidnightBlue}{\Large{\skt वरतर-रूप-विशेषैर् मारुत-दिग्-दल-निषण्ण-देहा या ।\\ 
पूजयति भैरवं तां इन्द्राणीं दृक्-तनुं वन्दे ॥}}}\\
I worship that \textit{Indrāṇī} of the form of sight, who worships the \textit{Bhairava} with excellent forms, with her body stationed in the petal of \textit{Vāyu}.\\

{\color{MidnightBlue}{\Large{\skt धनपति-किसलय-निलया या नित्यं विविध-षड्-रसा-हारैः ।\\ 
पूजयति भैरवं तां जिह्वाभिख्यां नमामि चामुण्डां ॥}}}\\
I salute that \textit{Cāmuṇḍā} of the form of the tongue, who ever worships the \textit{Bhairava} with the garlands of the diversity of six tastes, stationed in the petal of \textit{Kubera}.\\

{\color{MidnightBlue}{\Large{\skt ईश-दलस्था भैरवम् अर्चयते परिमलैर् विचित्रैर् या ।\\ 
प्रणमामि सर्वदा तां घ्राणाभिख्यां महालक्ष्मीम् ॥}}}\\
I ever salute that \textit{Mahālakṣmī} of the form of the nose, who worships the \textit{Bhairava} with diverse smells stationed in the petal of \textit{Īśāna}.\\

{\color{MidnightBlue}{\Large{\skt षड्-दर्शनेषु पूज्यं षट्-त्रिंशत्-तत्त्व-संवलितम् ।\\ 
आत्माभिख्यं सततं क्षेत्रपतिं सिद्धिदं वन्दे ॥}}}\\
I ever worship that success-giving \textit{Kṣetrapati} of the form of the first-person-experiencer worshiped in the six schools and encircled by the 36 \textit{tattva-s}.\\

{\color{MidnightBlue}{\Large{\skt संस्फुरद्-अनुभव-सारं सर्वान्तः सतत संनिहितम् ।\\
 नौमि सदोदितम् इत्थं निज-देहग-देवता-चक्रम् ॥}}}\\
I salute the ever-renewing circle of deities right here in my own body, vibrating in unison, in continual conjunction, present within all [existence] as the essence of first-person experience.
\end{document}